\documentclass[conference]{IEEEtran}
\IEEEoverridecommandlockouts
% The preceding line is only needed to identify funding in the first footnote. If that is unneeded, please comment it out.
\usepackage{lipsum}
\usepackage{cite}
\usepackage{amsmath,amssymb,amsfonts}
\usepackage{algorithmic}
\usepackage{graphicx}
\usepackage{textcomp}
\usepackage{xcolor}
\def\BibTeX{{\rm B\kern-.05em{\sc i\kern-.025em b}\kern-.08em
		T\kern-.1667em\lower.7ex\hbox{E}\kern-.125emX}}
	
\newbox\one
\newbox\two
\long\def\loremlines#1{%
	\setbox\one=\vbox {%
		\footnote{}%
		\lipsum\footnote{}%
	}
	\setbox\two=\vsplit\one to #1\baselineskip
	\unvbox\two}
	\usepackage[acronym]{glossaries}
	\newacronym{reps}{REPS}{Relational Event Prediction System}
\newacronym{gui}{GUI}{Graphical User Interface}
\begin{document}

	\title{Root-Causing and Event Identification Through Sensor Data}
	
	\author{\IEEEauthorblockN{1\textsuperscript{st} Simon dos Reis Spedsbjerg}
		\IEEEauthorblockA{\textit{MMMI} \\
			\textit{SDU}\\
			Odense, Denmark \\
			sispe20@student.sdu.dk}
	}
	
	\maketitle
	
	\begin{abstract}
		\loremlines{10}
	\end{abstract}
	
	\begin{IEEEkeywords}
		component, formatting, style, styling, insert
	\end{IEEEkeywords}
	
	\section{Introduction}	
		Modern machine learning systems has a certain level of uncertainty. Developers (depending on the model) cannot always explain how a model makes determinations.  Through statics and training, a model bases its decisions on patterns it has learned during its training. This makes it challenging for developers to understand how the system operates internally, verify the results, or ensure the verifiability of the outcomes. Additionally, many machine learning systems require a significant amount of data and extensive training before they can make reliable determinations.
		
		To address these challenges, I present \gls{reps}, a graph-based real-time analysis tool which developers themselves can design to fit their use case while making use of many state-of-the-art machine learning models. \gls{reps} operates by processing input data, which is likely will be sensor data which will pass through nodes called "event nodes". Each node processes inputs from one or more sources (a sensor or event node) and generates a result that can be used on one or multiple downstream nodes. This system enables	real-time decision-making across interconnected components.
		
		This project aims to determine whether there are any viable use cases for \gls{reps} and, if so, to what extent it performs relative to similar state-of-the-art solutions in terms of efficiency, accuracy, and scalability.
		
	\section{Context}
		For this project, some definitions is set to achieve best understanding of the project.
		\begin{itemize}
			\item \textbf{Change of State}, a change of state is when one or more of attributes of an object have changed. A change of state might be a concern that one would want to act on, i.e. planning reasons, mitigation or avoidance.
			\item \textbf{Event}, an event is the change of the state of an object, it has one or more inputs and  only one output.
			\item \textbf{Ongoing Event}, an event is considered ongoing if the change has not reversed to a normal state, the normal state being the state the object was before the event.
			\item \textbf{Value}, A value is produced by a sensor or an event.
			\item \textbf{Event Rule}, Event rules are logical statements or criteria that must be met before an event can occur. An event rule can be attached to one or more events, specifying the necessary conditions for their occurrence. An event can occur without the rules being fulfilled indicates a misalignment between the \gls{reps} view of the system and the system itself, similarly, if a rule is fulfilled but the event does not occur.
			\item \textbf{Trigger} An event is triggered when the change of state occurs. When an event is triggered, it emits a boolean value.
		\end{itemize}
		
	\section{Implementation}
		The \gls{reps} is split up in 2 parts, the system itself and then a \gls{gui}. 
		\subsection{REPS}\label{Reps:Imp}
			It starts of loading a config file, this file is in a form of JSON. JSON was used due to how easy it is to modify between tests as well as it being a type which can function with other systems such as the \gls{gui}. It will create the different nodes from this JSON config, these configs are structs containing all of the data for their respective nodes. A sensor node receives data from a sensor or through another communication method. The nodes makes use of publish subscribe pattern, and I used RabbitMQ for all communication due to it supporting priority queuing. A Event node retrieves the data from the sensor nodes to then use in their model, the output of this model is then made available to other nodes. A Event node also has a Trigger function which is used to determine severity of the ongoing event.
			
			Each Node will make a connection to the MQTT broker with a specific topic which it either publishes to or retrieves data from.
			
			
			\subsubsection{Models}
				Models uses C\# Rosalyns compiler to compile code during run time, this will allow for a custom functions to make determinations. Using Rosalyns compiler during runtime allow for full customization without have the need to implement own interpreter. This increases the complexity for the user as they have to write their function in a C\# syntax and it gives a risk of arbitrary code execution.
				
				
				Each Event node contains a model which it uses to determine a output, the model can vary between their complexity, a simple model can take a math function in a C\# syntax and generate an output from that.
				
				%TODO: Beskriv andre modeller her.
			\subsubsection{Trigger}
				A trigger is implemented much the same way as a simple model, the difference is that the trigger i more limited in what it can return, this is because there is not the need for it and this limits arbitrary code execution.
			
			\subsubsection{Misc}
				Each node can run within its own threads, avoiding the problem of them having to wait for one another, the problem with this is the faster nodes may reuse data from other nodes which hasn't finished. This is accepted thinking that each data point is not necessary to determine events.
	
		\subsection{\gls{gui}}
			The \gls{gui} is developed to help during testing and experimentation. It is just as much a development tool as it is part of the final product for the user. The \gls{gui} is developed in C\#'s blazor as it supports linux platform which other C\# platforms doesn't. It being C\# allows for the system to use the config files implemented in the REPS mentioned in section \ref{Reps:Imp}. As it is a development tool I do not intend to make updates to it when I update the REPS, so having it dynamically change depending on config files will ease development and testing.
			
			Blazor was also chosen for being component based, I can easily add new pages or reuse components if the need arises.
			
			%TODO: Hvis jeg formår at lave det, skriv om visning af noderne i grafisk her.
	
	\section{Lipsum}
		\loremlines{50}
	
	\subsection{Authors and Affiliations}
		\loremlines{15}

	
	\section{Acknowledgment}
		\loremlines{10}
	
	\section*{References}
		\loremlines{20}
	
	
\end{document}